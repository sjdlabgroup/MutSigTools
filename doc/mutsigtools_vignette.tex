\documentclass[]{article}
\usepackage{lmodern}
\usepackage{amssymb,amsmath}
\usepackage{ifxetex,ifluatex}
\usepackage{fixltx2e} % provides \textsubscript
\ifnum 0\ifxetex 1\fi\ifluatex 1\fi=0 % if pdftex
  \usepackage[T1]{fontenc}
  \usepackage[utf8]{inputenc}
\else % if luatex or xelatex
  \ifxetex
    \usepackage{mathspec}
  \else
    \usepackage{fontspec}
  \fi
  \defaultfontfeatures{Ligatures=TeX,Scale=MatchLowercase}
\fi
% use upquote if available, for straight quotes in verbatim environments
\IfFileExists{upquote.sty}{\usepackage{upquote}}{}
% use microtype if available
\IfFileExists{microtype.sty}{%
\usepackage{microtype}
\UseMicrotypeSet[protrusion]{basicmath} % disable protrusion for tt fonts
}{}
\usepackage[margin=1in]{geometry}
\usepackage{hyperref}
\PassOptionsToPackage{usenames,dvipsnames}{color} % color is loaded by hyperref
\hypersetup{unicode=true,
            pdftitle={Package MutSigTools},
            pdfauthor={Vinod Singh, Neil Coleman and Subhajyoti De},
            colorlinks=true,
            linkcolor=blue,
            citecolor=Blue,
            urlcolor=Blue,
            breaklinks=true}
\urlstyle{same}  % don't use monospace font for urls
\usepackage{natbib}
\bibliographystyle{plainnat}
\usepackage{color}
\usepackage{fancyvrb}
\newcommand{\VerbBar}{|}
\newcommand{\VERB}{\Verb[commandchars=\\\{\}]}
\DefineVerbatimEnvironment{Highlighting}{Verbatim}{commandchars=\\\{\}}
% Add ',fontsize=\small' for more characters per line
\usepackage{framed}
\definecolor{shadecolor}{RGB}{248,248,248}
\newenvironment{Shaded}{\begin{snugshade}}{\end{snugshade}}
\newcommand{\KeywordTok}[1]{\textcolor[rgb]{0.13,0.29,0.53}{\textbf{#1}}}
\newcommand{\DataTypeTok}[1]{\textcolor[rgb]{0.13,0.29,0.53}{#1}}
\newcommand{\DecValTok}[1]{\textcolor[rgb]{0.00,0.00,0.81}{#1}}
\newcommand{\BaseNTok}[1]{\textcolor[rgb]{0.00,0.00,0.81}{#1}}
\newcommand{\FloatTok}[1]{\textcolor[rgb]{0.00,0.00,0.81}{#1}}
\newcommand{\ConstantTok}[1]{\textcolor[rgb]{0.00,0.00,0.00}{#1}}
\newcommand{\CharTok}[1]{\textcolor[rgb]{0.31,0.60,0.02}{#1}}
\newcommand{\SpecialCharTok}[1]{\textcolor[rgb]{0.00,0.00,0.00}{#1}}
\newcommand{\StringTok}[1]{\textcolor[rgb]{0.31,0.60,0.02}{#1}}
\newcommand{\VerbatimStringTok}[1]{\textcolor[rgb]{0.31,0.60,0.02}{#1}}
\newcommand{\SpecialStringTok}[1]{\textcolor[rgb]{0.31,0.60,0.02}{#1}}
\newcommand{\ImportTok}[1]{#1}
\newcommand{\CommentTok}[1]{\textcolor[rgb]{0.56,0.35,0.01}{\textit{#1}}}
\newcommand{\DocumentationTok}[1]{\textcolor[rgb]{0.56,0.35,0.01}{\textbf{\textit{#1}}}}
\newcommand{\AnnotationTok}[1]{\textcolor[rgb]{0.56,0.35,0.01}{\textbf{\textit{#1}}}}
\newcommand{\CommentVarTok}[1]{\textcolor[rgb]{0.56,0.35,0.01}{\textbf{\textit{#1}}}}
\newcommand{\OtherTok}[1]{\textcolor[rgb]{0.56,0.35,0.01}{#1}}
\newcommand{\FunctionTok}[1]{\textcolor[rgb]{0.00,0.00,0.00}{#1}}
\newcommand{\VariableTok}[1]{\textcolor[rgb]{0.00,0.00,0.00}{#1}}
\newcommand{\ControlFlowTok}[1]{\textcolor[rgb]{0.13,0.29,0.53}{\textbf{#1}}}
\newcommand{\OperatorTok}[1]{\textcolor[rgb]{0.81,0.36,0.00}{\textbf{#1}}}
\newcommand{\BuiltInTok}[1]{#1}
\newcommand{\ExtensionTok}[1]{#1}
\newcommand{\PreprocessorTok}[1]{\textcolor[rgb]{0.56,0.35,0.01}{\textit{#1}}}
\newcommand{\AttributeTok}[1]{\textcolor[rgb]{0.77,0.63,0.00}{#1}}
\newcommand{\RegionMarkerTok}[1]{#1}
\newcommand{\InformationTok}[1]{\textcolor[rgb]{0.56,0.35,0.01}{\textbf{\textit{#1}}}}
\newcommand{\WarningTok}[1]{\textcolor[rgb]{0.56,0.35,0.01}{\textbf{\textit{#1}}}}
\newcommand{\AlertTok}[1]{\textcolor[rgb]{0.94,0.16,0.16}{#1}}
\newcommand{\ErrorTok}[1]{\textcolor[rgb]{0.64,0.00,0.00}{\textbf{#1}}}
\newcommand{\NormalTok}[1]{#1}
\usepackage{graphicx,grffile}

\usepackage{makeidx}
\makeindex

\makeatletter
\def\maxwidth{\ifdim\Gin@nat@width>\linewidth\linewidth\else\Gin@nat@width\fi}
\def\maxheight{\ifdim\Gin@nat@height>\textheight\textheight\else\Gin@nat@height\fi}
\makeatother
% Scale images if necessary, so that they will not overflow the page
% margins by default, and it is still possible to overwrite the defaults
% using explicit options in \includegraphics[width, height, ...]{}
\setkeys{Gin}{width=\maxwidth,height=\maxheight,keepaspectratio}
\IfFileExists{parskip.sty}{%
\usepackage{parskip}
}{% else
\setlength{\parindent}{0pt}
\setlength{\parskip}{6pt plus 2pt minus 1pt}
}
\setlength{\emergencystretch}{3em}  % prevent overfull lines
\providecommand{\tightlist}{%
  \setlength{\itemsep}{0pt}\setlength{\parskip}{0pt}}
\setcounter{secnumdepth}{0}
% Redefines (sub)paragraphs to behave more like sections
\ifx\paragraph\undefined\else
\let\oldparagraph\paragraph
\renewcommand{\paragraph}[1]{\oldparagraph{#1}\mbox{}}
\fi
\ifx\subparagraph\undefined\else
\let\oldsubparagraph\subparagraph
\renewcommand{\subparagraph}[1]{\oldsubparagraph{#1}\mbox{}}
\fi

%%% Use protect on footnotes to avoid problems with footnotes in titles
\let\rmarkdownfootnote\footnote%
\def\footnote{\protect\rmarkdownfootnote}

%%% Change title format to be more compact
\usepackage{titling}

% Create subtitle command for use in maketitle
\newcommand{\subtitle}[1]{
  \posttitle{
    \begin{center}\large#1\end{center}
    }
}

\setlength{\droptitle}{-2em}

  \title{Package `MutSigTools'}
    \pretitle{\vspace{\droptitle}\centering\huge}
  \posttitle{\par}
    \author{Vinod Singh, Neil Coleman and Subhajyoti De}
    \preauthor{\centering\large\emph}
  \postauthor{\par}
      \predate{\centering\large\emph}
  \postdate{\par}
    \date{2019-02-22}

\usepackage{setspace}\singlespacing
\usepackage{float}

\begin{document}
\maketitle

\renewcommand{\abstract}{}

\begin{abstract}
\doublespacing

\noindent

\textbf{Title}: MutSigTools\\
\textbf{Type}: Package\\
\textbf{Version}: 1.0.0\\
\textbf{Description}: MutSigTools implements a wide range of generic
utilities for processing and analysis of mutation signatures.\\
\textbf{Author}: Vinod Kumar Singh, Neil Coleman and Subhajyoti De\\
\textbf{Maintainer}: The package maintainer
\href{mailto:vs580@scarletmail.rutgers.edu}{\nolinkurl{vs580@scarletmail.rutgers.edu}}\\
\textbf{License}:  MIT + file LICENSE\\
\textbf{Encoding}: UTF-8\\
\textbf{LazyData}: true\\
\textbf{Depends}: R (\textgreater{}= 3.4.0), bedr, gplots, factoextra,
data.table, tools, stats, GenomicRanges, Biostrings, deconstructSigs.\\
\textbf{RoxygenNote}: 6.1.1\\
\textbf{Suggests}:devtools, knitr, rmarkdown.\\
\textbf{VignetteBuilder}: knitr.\\
\textbf{URL}: \url{https://github.com/sjdlabgroup/MutSigTools}\\
\textbf{BugReports}: \url{https://github.com/sjdlabgroup/MutSigTools/issues}
\end{abstract}

\newpage

\singlespacing
\renewcommand{\contentsname}{R topics documented:}

\tableofcontents


\DeclareRobustCommand{\bs}[1]{\texttt{#1}\index{BSgenome!#1}}
\DeclareRobustCommand{\dfs}[1]{\texttt{#1}\index{Dataframes!#1}}
\DeclareRobustCommand{\sigs}[1]{\texttt{#1}\index{Signatures!#1}}
\DeclareRobustCommand{\var}[1]{\texttt{#1}\index{Variation among Signatures!#1}}


\subsection{1. Introduction:}\label{introduction}

The patterns of mutation accumulation in tumor genomes provide insights
into past exposure to mutagens, mechanism of DNA damage, repair defects,
and the extent of genomic instability during development, aging, and
cancer. A number of mutation signatures were initially identified using
non-negative matrix-factorization-based approaches in analyzing cancer
genomic data, providing insights into etiology of the malignant diseases
\citet{alexandrov2013signatures}, \citet{alexandrov2014mutational}.
Subsequently additional mutation signatures have been identified using
experimental and computational approaches \citet{boot2018depth}
MutSigTools implements wide range of utilities for processing, analysis,
and interpretation of somatic mutations and associated mutation
signatures.

\subsection{2. Availability and
Installation}\label{availability-and-installation}

The development version of \texttt{MutSigTools} package is available at
\url{https://github.com/sjdlabgroup/MutSigTools} and can be installed as

\begin{Shaded}
\begin{Highlighting}[]
\CommentTok{# install.packages("devtools")}
\NormalTok{devtools}\OperatorTok{::}\KeywordTok{install_github}\NormalTok{(}\StringTok{"sjdlabgroup/MutSigTools"}\NormalTok{,}\DataTypeTok{build_vignettes =} \OtherTok{TRUE}\NormalTok{ )}
\end{Highlighting}
\end{Shaded}

\subsection{3. Functions}\label{functions}

\hypertarget{vcfToSNV}{\subsubsection{3.1 vcfToSNV}\label{vcfToSNV}}

\paragraph{\texorpdfstring{\textbf{Description}}{Description}}\label{description}

Reads a \texttt{vcf} formatted file to catalog mutations in a data frame
of class \texttt{snv} for downstream analysis. A \dfs{snv} dataframe consists of
\texttt{sample,\ chr,\ pos,\ ref,\ alt\ and/or\ freq} columns.

\paragraph{\texorpdfstring{\textbf{Usage}}{Usage}}\label{usage}

\texttt{vcfToSNV(vcf,\ allelefreq\ =\ FALSE)}

\paragraph{\texorpdfstring{\textbf{Arguments
}}{Arguments }}\label{arguments}

\begin{itemize}
\tightlist
\item
  \texttt{vcf}: The \texttt{path} of a standard \texttt{vcf\ file}.
\item
  \texttt{allelefreq}: A logical input, TRUE if mutation allele
  frequency is required in output object. \texttt{Default:\ FALSE}
\end{itemize}

\paragraph{\texorpdfstring{\textbf{Details}}{Details}}\label{details}

Read \texttt{vcf} formatted file to catalog mutations in a data frame of class
\texttt{snv} for downstream analysis for each input sample.

\hypertarget{snv}{\paragraph{\texorpdfstring{\textbf{Value}}{Value}}\label{snv}}

\begin{enumerate}
\def\labelenumi{\arabic{enumi}.}
\tightlist
\item
  \texttt{snv} is a dataframe with columns
  \texttt{sample,\ chr,\ pos,\ ref,\ alt,\ and/or\ freq}. Where,

  \begin{itemize}
  \tightlist
  \item
    \texttt{sample} : Name of the sample name.
  \item
    \texttt{chr} : chromosome number
  \item
    \texttt{pos} : position of mutation
  \item
    \texttt{ref} : Reference base (mutated from)
  \item
    \texttt{alt} : mutated base (mutated to)
  \item
    \texttt{freq} : Alt Allele Frequency (optional)
  \end{itemize}
\end{enumerate}

Note: \texttt{vcf\ file\ name} with no extension will be the
\texttt{sample\ name}.

\paragraph{\texorpdfstring{\textbf{Examples}}{Examples}}\label{examples}

\begin{verbatim}
> vcf_file=system.file("extdata", "test.vcf", package = "MutSigTools", mustWork = TRUE) 
> snv=vcfToSNV(vcf=vcf_file, allelefreq=TRUE)
> head(snv)
     sample  chr   pos ref alt     freq
1 CRF004556 chr1 15118   A   G 1.000000
2 CRF004556 chr1 15211   T   G 0.935484
3 CRF004556 chr1 15237   C   T 0.153846
4 CRF004556 chr1 16497   A   G 0.833333
5 CRF004556 chr1 16534   C   T 1.000000
6 CRF004556 chr1 16571   G   A 1.000000
\end{verbatim}

\subsubsection{3.2 contextSNV}\label{contextsnv}

\paragraph{\texorpdfstring{\textbf{Description}}{Description}}\label{description-1}

Group mutations according to their occurrence within or outside
user-annotated genomic segments.

\paragraph{\texorpdfstring{\textbf{Usage}}{Usage}}\label{usage-1}

\texttt{contextSNV(snv,\ file\ =\ "empty",\ mode\ =\ "include", \ bed\_file\_start\_cord \ = \ 0)}

\paragraph{\texorpdfstring{\textbf{Arguments
}}{Arguments }}\label{arguments-1}

\begin{itemize}
\tightlist
\item
  \protect\hyperlink{snv}{\texttt{snv}}: A dataframe having
  \texttt{sample,\ chr,\ pos,\ ref,\ alt\ and/or\ freq} as its columns.
  This snv dataframe can be created by the
  \protect\hyperlink{vcfToSNV}{vcfToSNV} function of this package.
\item
  \texttt{file}: Path of the \texttt{BED\ file}, having genomic segments
  and their context information.
\item
  \texttt{mode}: Default: \texttt{include}: Select mutations lying
  inside the given genomic contexts (obtained from BED file).
  \texttt{exclude}: will give mutations outside the given context.
\item
  \texttt{bed\_file\_start\_cord}: Default: \texttt{0}: If BED file has 0-based coordinate system. \texttt{1}: If BED file has 1-based coordinate system..
  
\end{itemize}

\paragraph{\texorpdfstring{\textbf{Details}}{Details}}\label{details-1}

When the \texttt{include} mode is used, group mutations according to
their occurrence in user-specified annotated genomic segments; such
segments could represent specific genomic or epigenomic contexts
(e.g.~actively transcribed coding regions). Alternately, in the
\texttt{exclude} mode, mutations in undesired regions (e.g.black-listed
genomic regions) can be removed.

\paragraph{\texorpdfstring{\textbf{Value}}{Value}}\label{value}

\texttt{snv} dataframe with column names of
\texttt{sample,\ chr,\ pos,\ ref,\ alt,\ freq}. Where, sample name will
be concatenated by the its genomic context. Example: if a mutation
within a sample of name \texttt{sample1}, lies in the context segment of
label \texttt{C1} given in the input \texttt{bed\ file} , the sample
name of mutation is modified to \texttt{sample1\_C1}. The mutations that
don't lie in any context segment given in the \texttt{bed\ file} are
discarded.

\paragraph{\texorpdfstring{\textbf{Examples}}{Examples}}\label{examples-1}

\begin{verbatim}
> BED_file=system.file("extdata", "context_testFile.bed", package = "MutSigTools", mustWork = TRUE)
> data(snv_sample)   # load 'snv' dataframe object
> context_snv=contextSNV(snv=snv_sample,BED_file, mode='include', bed_file_start_cord=0)
> head(context_snv)
        sample  chr   pos ref alt
1 CRF004556_E7 chr1 15118   A   G
2 CRF004556_E7 chr1 15211   T   G
3 CRF004556_E7 chr1 15237   C   T
4 CRF004556_E7 chr1 16497   A   G
5 CRF004556_E7 chr1 16534   C   T
6 CRF004556_E7 chr1 16571   G   A
\end{verbatim}

\subsubsection{3.3 processSNV}\label{processsnv}

\paragraph{\texorpdfstring{\textbf{Description}}{Description}}\label{description-2}

Estimate frequency of point mutations in tri-nucleotide motif contexts for one or more input samples.


\paragraph{\texorpdfstring{\textbf{Usage}}{Usage}}\label{usage-2}

\texttt{processSNV(snv,\ bsg\ =\ \bs{BSgenome.Hsapiens.UCSC.hg19}::Hsapiens)}

\paragraph{\texorpdfstring{\textbf{Arguments
}}{Arguments }}\label{arguments-2}

\begin{itemize}
\tightlist
\item
  \protect\hyperlink{snv}{\texttt{snv}}: A dataframe having sample, chr,
  pos, ref, alt and/or freq as its columns. The \texttt{snv} dataframe
  can be created by \protect\hyperlink{vcfToSNV}{\texttt{vcfToSNV}}
  function.
\item
  \texttt{bsg}: An boject of class BSGenome. \texttt{Default:}
  \href{https://bioconductor.org/packages/release/data/annotation/html/BSgenome.Hsapiens.UCSC.hg19.html}{BSgenome.Hsapiens.UCSC.hg19::Hsapiens}
\end{itemize}

\paragraph{\texorpdfstring{\textbf{Details}}{Details}}\label{details-2}

Estimation of mutation frequency in pre-determined, user-specified
nucleotide motif contexts for input samples. The \bs{BSgenome} object
contains the reference genome information, and should be specified for
alternative assembly of human genomes or nonhuman reference genomes. The
output returns a data frame containing estimated mutation frequencies in
the different genomic contexts, as provided in the \dfs{\texttt{contextfreq}}
object for each input sample.

\hypertarget{contextfreq}{\paragraph{\texorpdfstring{\textbf{Value}}{Value}}\label{contextfreq}}

A data frame of class \texttt{contextfreq} containing mutation frequency
in user-specified nucleotide contexts.

\paragraph{\texorpdfstring{\textbf{Examples}}{Examples}}\label{examples-2}

\begin{verbatim}
> data(snv_sample)
> context.freq=processSNV(snv=snv_sample, bsg = BSgenome.Hsapiens.UCSC.hg19::Hsapiens)
> context.freq[,1:6]   # view first 6 mutation frequency contexts of 96
          A[C>A]A A[C>A]C A[C>A]G A[C>A]T C[C>A]A C[C>A]C
CRF004556    1920    1412     653    1180    1845    1439
\end{verbatim}

\subsubsection{3.4 addSignature}\label{addsignature}

\paragraph{\texorpdfstring{\textbf{Description}}{Description}}\label{description-3}

Add new signature(s) to an existing set of \sigs{mutation signatures}, and
return the updated signature set.

\paragraph{\texorpdfstring{\textbf{Usage}}{Usage}}\label{usage-3}

\texttt{addSignature(sigmatrix2,\ sigmatrix1)}

\paragraph{\texorpdfstring{\textbf{Arguments
}}{Arguments }}\label{mutsig}

\begin{itemize}
\tightlist
\item
  \texttt{\dfs{sigmatrix}1}: An object of class \texttt{mutsig} describing the
  existing set of signatures.
\item
  \texttt{sigmatrix2}: An object of class \texttt{mutsig} describing the
  new signature(s).
\end{itemize}

\paragraph{\texorpdfstring{\textbf{Value}}{Value}}\label{value-1}

The list of mutational signatures with the new signature added.

\paragraph{\texorpdfstring{\textbf{Examples}}{Examples}}\label{examples-3}

\begin{verbatim}
> newSigMatrix=addSignature(sigmatrix2, sigmatrix1)
\end{verbatim}

\subsubsection{3.5 deleteSignature}\label{deletesignature}

\paragraph{\texorpdfstring{\textbf{Description}}{Description}}\label{description-4}

Delete selected signatures from an existing set of mutation signatures.

\paragraph{\texorpdfstring{\textbf{Usage}}{Usage}}\label{usage-4}

\texttt{deleteSignature(sigmatrix,\ del\_sig)}

\paragraph{\texorpdfstring{\textbf{Arguments
}}{Arguments }}\label{arguments-3}

\begin{itemize}
\tightlist
\item
  \texttt{sigmatrix}: An object of class \texttt{mutsig} describing the
  existing set of signatures.
\item
  \texttt{del\_sig}: Signature(s) to be removed.
\end{itemize}

\subparagraph{\texorpdfstring{\textbf{Details}}{Details}}\label{details-3}

Deletion of one or more mutation signatures from an existing set of
mutation signatures upon which the updated signature set is returned.

\paragraph{\texorpdfstring{\textbf{Value}}{Value}}\label{value-2}

An object of mutsig class with the deleted signatures.

\paragraph{\texorpdfstring{\textbf{Examples}}{Examples}}\label{examples-4}

\begin{verbatim}
> reducedSigMatrix=deleteSignature(sigmatrix=signatures.cosmic, c(2,5)) # delete signature 2 and 5.
\end{verbatim}

\subsubsection{3.6 renameSignature}\label{renamesignature}

\paragraph{\texorpdfstring{\textbf{Description}}{Description}}\label{description-5}

Rename selected signatures from an existing set of signatures.

\paragraph{\texorpdfstring{\textbf{Usage}}{Usage}}\label{usage-5}

\texttt{renameSignature(sigmatrix,selectSig,renameSig)}

\paragraph{\texorpdfstring{\textbf{Arguments
}}{Arguments }}\label{arguments-4}

\begin{itemize}
\tightlist
\item
  \texttt{sigmatrix}: An object of class mutsig describing the existing
  set of signatures.
\item
  \texttt{selectSig}: A vector containing name(s) of the signature(s) to
  be removed.
\item
  \texttt{renameSig}: A vector containing updated name(s) of the
  signatures.
\end{itemize}

\paragraph{\texorpdfstring{\textbf{Details}}{Details}}\label{details-4}

Renaming of selected mutation signatures to an existing signature set,
upon which the updated signature set is returned.

\paragraph{\texorpdfstring{\textbf{Value}}{Value}}\label{value-3}

An object of \texttt{mutsig} class.

\paragraph{\texorpdfstring{\textbf{Examples}}{Examples}}\label{examples-5}

\begin{verbatim}
> renamedSigMatrix=renameSignature(sigmatrix=signatures.cosmic, selectSig=sig=c(2,5), 
> renameSig=resig=c('two', 'five')) # rename signatures 2 and 5.
\end{verbatim}

\subsubsection{3.7 mergeSignature}\label{mergesignature}

\subparagraph{\texorpdfstring{\textbf{Description}}{Description}}\label{description-6}

This function creates a new mutational signature by taking a linear
combination of \texttt{k} other signatures s\textsubscript{11},
s\textsubscript{12},\ldots{},s\textsubscript{kn} in a matrix of
mutational signatures. In order to use this function, a weight vector
w\textsubscript{1}, w\textsubscript{2},\ldots{},w\textsubscript{k} must
be specified. The new mutational signature is created by multiplying the
component signature values by the corresponding weights in the weight
vector. The formula for the entries of the new signature is
s\textsubscript{11}w\textsubscript{1}+s\textsubscript{21}w\textsubscript{2}+\ldots{}+s\textsubscript{k1}w\textsubscript{k},
s\textsubscript{12}w\textsubscript{1}+s\textsubscript{22}w\textsubscript{2}+\ldots{}+s\textsubscript{k2}w\textsubscript{k},\ldots{},s\textsubscript{1n}w\textsubscript{1}+s\textsubscript{2n}w\textsubscript{2}+\ldots{}+s\textsubscript{kn}w\textsubscript{k}.
The function returns this new signature plus all of the signatures in
the original dataset that were not merged into the new matrix of
signatures.

\paragraph{\texorpdfstring{\textbf{Usage}}{Usage}}\label{usage-6}

\texttt{mergeSignature(sigmatrix,sig,weights)}

\paragraph{\texorpdfstring{\textbf{Arguments
}}{Arguments }}\label{arguments-5}

\begin{itemize}
\tightlist
\item
  \texttt{sigmatrix}: A \texttt{mutsig} object of mutational signatures.
\item
  \texttt{sig}: A vector that describes the signatures to be merged
\item
  \texttt{weights}: A vector that describes the relative weights of each
  of the component signatures in the merged mutational signature.
\end{itemize}

\paragraph{\texorpdfstring{\textbf{Value}}{Value}}\label{value-4}

A new matrix of signatures consisting of the signatures in the original
matrix of mutational signatures that were not merged as well as the
mutational signature that is a result of merging the input signatures.
For example, if there were 10 signatures in the original signature
matrix and signatures 1-6 were set to be merged, then the output value
would be a matrix of 5 signatures(signature 7,8,9,10 and the result of
merging signatures 1-6).

\paragraph{\texorpdfstring{\textbf{Examples}}{Examples}}\label{examples-6}

\begin{verbatim}
> weights=rep(1/27,27)      
> mergeSignature(signatures.cosmic,1:27,weights)
> # another example
> weights=rep(1/6,6)  
> mergeSignature(signatures.cosmic,1:6,weights)   
\end{verbatim}

\subsubsection{3.8 clusterSignature}\label{clustersignature}

Performs a \var{hierarchical clustering} analysis of an input mutational
signature matrix using the Euclidean distance metric.

\paragraph{\texorpdfstring{\textbf{Usage}}{Usage}}\label{usage-7}

\texttt{clusterSignature(sigmatrix)}

\paragraph{\texorpdfstring{\textbf{Arguments
}}{Arguments }}\label{arguments-6}

\begin{itemize}
\tightlist
\item
  \texttt{sigmatrix}: An input matrix of mutational signatures.
\end{itemize}

\paragraph{\texorpdfstring{\textbf{Value}}{Value}}\label{value-5}

A hierarchical clustering object formed from the input matrix of
mutational signatures. This can then be plotted to show how the
mutational signatures in the input matrix are related to one another.

\paragraph{\texorpdfstring{\textbf{Examples}}{Examples}}\label{examples-7}

\begin{verbatim}
> hclust_obj=clusterSignature(sigmatrix=signatures.cosmic) 
\end{verbatim}

\subsubsection{3.9 signatureHeatmap}\label{signatureheatmap}

\paragraph{\texorpdfstring{\textbf{Description}}{Description}}\label{description-7}

Construct a \var{heatmap} showing relative weights of tri-nucleotide motif for
different mutation signatures.

\paragraph{\texorpdfstring{\textbf{Usage}}{Usage}}\label{usage-8}

\texttt{signatureHeatmap(sigmat,\ pngfile=\textquotesingle{}signatures\_heatmap\textquotesingle{}\ )}

\paragraph{\texorpdfstring{\textbf{Arguments
}}{Arguments }}\label{arguments-7}

\begin{itemize}
\tightlist
\item
  \texttt{sigmat}: An object of class \texttt{mutsig} describing the
  existing set of signatures.
\item
  \texttt{pngfile}: The name of \texttt{png} formatted image file to be
  created.
\end{itemize}

\paragraph{\texorpdfstring{\textbf{Details}}{Details}}\label{details-5}

Create a heatmap to show relative weights of trinucleotide motifs for
different mutation signatures in a signature-set.

\paragraph{\texorpdfstring{\textbf{Value}}{Value}}\label{value-6}

A png formatted image file.

\paragraph{\texorpdfstring{\textbf{Examples}}{Examples}}\label{examples-8}

\begin{verbatim}
> sigmat=signatures.cosmic
> signatureHeatmap(sigmat, pngfile=‘signatures_heatmap’)
\end{verbatim}

\subsubsection{3.10 signaturePCA}\label{signaturepca}

\paragraph{\texorpdfstring{\textbf{Description}}{Description}}\label{description-8}

Constructs three PCA plots showing variations among the signatures in
terms of the weights of the principal components for different mutation
signatures, as well as the eigenvalues and cosine similarity among
expected and actual signatures.

\paragraph{\texorpdfstring{\textbf{Usage}}{Usage}}\label{usage-9}

\texttt{\var{signaturePCA}(sigmat,\ pngfile\ )}

\paragraph{\texorpdfstring{\textbf{Arguments
}}{Arguments }}\label{arguments-8}

\begin{itemize}
\tightlist
\item
  \texttt{sigmat}: An object of class \texttt{mutsig} describing a set
  of signatures.
\item
  \texttt{pngfile}: Name of \texttt{png} formatted image file to be
  created.
\end{itemize}

\paragraph{\texorpdfstring{\textbf{Value}}{Value}}\label{value-7}

Create three \texttt{PCA\ plots} with the following extensions to show
difference among mutational signatures.

\begin{itemize}
\tightlist
\item
  \texttt{*\_Eigen.png} : The eigenvalues of the PCA and the relative
  amount of the variation each eigenvalue.
\item
  \texttt{*\_var.png} : The different single-nucleotide alteration
  components in terms of the two dimensions.
\item
  \texttt{*\_PCA.png} : The different signatures in terms of the 2
  non-dimensional vectors derived by the PCA analysis and the cosine
  similarity between the estimate of each mutation signature using PCA
  and the actual mutation signature(in order to determine how accurately
  the 2 dimensions calculated by PCA represent the mutational
  signatures).
\end{itemize}

\paragraph{\texorpdfstring{\textbf{Examples}}{Examples}}\label{examples-9}

\begin{verbatim}
> sigmat=signatures.cosmic
> signaturePCA(sigmat, pngfile=“signaturePCA”)
\end{verbatim}


\subsubsection{3.11 signature\_tSNE}\label{signaturetsne}

\paragraph{\texorpdfstring{\textbf{Description}}{Description}}\label{description-8}

Constructs tSNE plot of mutation signatures.

\paragraph{\texorpdfstring{\textbf{Usage}}{Usage}}\label{usage-9}

\texttt{\var{signature\_tSNE}(sigmat,\ pngfile\ )}

\paragraph{\texorpdfstring{\textbf{Arguments
}}{Arguments }}\label{arguments-8}

\begin{itemize}
\tightlist
\item
  \texttt{sigmat}: An object of class \texttt{mutsig} describing a set
  of signatures.
\item
  \texttt{pngfile}: Name of \texttt{png} formatted image file to be
  created.
\end{itemize}

\paragraph{\texorpdfstring{\textbf{Value}}{Value}}\label{value-7}

Create a \texttt{tSNE\ plots} plot to show similarity among mutational signatures.

\begin{itemize}
\tightlist
\item
  \texttt{*\_tSNE.png} : The tSNE plot file of the mutation signatures. 

\end{itemize}

\paragraph{\texorpdfstring{\textbf{Examples}}{Examples}}\label{examples-9}

\begin{verbatim}
> sigmat=signatures.cosmic
> signature_tSNE(sigmat, pngfile=“signature_tSNE”)
\end{verbatim}

\subsubsection{3.12 confidenceSig}\label{confidencesig}

\paragraph{\texorpdfstring{\textbf{Description}}{Description}}\label{description-9}

Provides an interval of uncertainty for estimated weights of known mutation signatures in the catalog of mutations from a set of samples.

\paragraph{\texorpdfstring{\textbf{Usage}}{Usage}}\label{usage-10}

\texttt{confidenceSig(contextfreq.sample,\ subsample=0.8,\ iter=1000\ ,\ signatures.ref=\sigs{signatures.cosmic},\ lbound=0.1,\ ubound=0.9, \ replace=FALSE)}

\paragraph{\texorpdfstring{\textbf{Arguments
}}{Arguments }}\label{arguments-9}

\begin{itemize}
\tightlist
\item
  \texttt{contextfreq.sample}: A sample from the dataframe of class
  \protect\hyperlink{contextfreq}{\texttt{contextfreq}} containing
  mutation frequency in tri-nucleotide contexts.
\item
  \texttt{subsample}: Proportion of mutations included during each
  subsampling. \texttt{Default:\ 0.8} (80 percent)
\item
  \texttt{iter}: Number of iterations of subsampling.
  \texttt{Default:\ 1000}
\item
  \texttt{signatures.ref}: An object of class mutsig comprising the set of signatures. (`signatures.nature2013' or `signatures.cosmic' or `signatures.cosmic.2019' ), Default: \href{https://cran.r-project.org/web/packages/deconstructSigs/deconstructSigs.pdf}{signatures.cosmic} 
\item
  \texttt{lbound}: Lower bound of the interval of uncertainty for
  estimated weights of the signatures. \texttt{Default:\ 0.1} (10
  percent)
\item
  \texttt{ubound}: Upper bound of the interval of uncertainty for
  estimated weights of the signatures. \texttt{Default:\ 0.9} (90
  percent)
\item
  \texttt{replace}: should sampling be with replacement? TRUE or FALSE. \texttt{Default:\ FALSE} 
\end{itemize}

\paragraph{\texorpdfstring{\textbf{Details}}{Details}}\label{details-6}

Provides an interval of uncertainty for estimated weights of known
mutation signatures in the catalog of mutations from a set of samples.
First, based on the catalog of mutations in a sample, weights of the
mutation signatures are estimated. Next, mutations are sub-sampled
iteratively without replacement, each time estimating the weights of the
mutation signatures. The intervals of uncertainty for weights of each
mutation signatures are determined by aggregating observations from a
given number of iterations.

\paragraph{\texorpdfstring{\textbf{Value}}{Value}}\label{value-8}

An object containing the following information:

\begin{itemize}
\tightlist
\item
  \texttt{observed.weights} : A data frame containing estimated weights
  of known mutation signatures based on all mutations in a sample.
\item
  \texttt{median.weights} : A data frame containing estimated median of
  the weights of known mutation signatures based on iteratively
  subsampled mutations.
\item
  \texttt{ubound.weights} : A data frame containing upper-bound values
  of the weights of known mutation signatures based on iteratively
  subsampled mutations.
\item
  \texttt{lbound.weights}: : A data frame containing lower-bound values
  of the weights of known mutation signatures based on iteratively
  subsampled mutations.
\end{itemize}

\paragraph{\texorpdfstring{\textbf{Examples}}{Examples}}\label{examples-10}

\begin{verbatim}
> data(contextfreq.sample_test)
> robust_sig_object=confidenceSig(contextfreq.sample=contextfreq.sample_test, subsample=0.8, iter=50, 
signatures.ref=signatures.cosmic, lbound=0.1, ubound=0.9, replace=FALSE)
\end{verbatim}

\subsubsection{3.13 persistSig}\label{persistsig}

Determine the burden of different mutation signatures across different allele frequency ranges. The function should be used with caution; Local copy number and purity-corrected variant allele frequencies should be provided in the column 6 of the \protect\hyperlink{snv}{\texttt{snv}} object.

\paragraph{\texorpdfstring{\textbf{Usage}}{Usage}}\label{usage-11}

\texttt{persistSig(snv,\ th\_vec\_lw,\ th\_vec\_up,\ bsg\ =\ BSgenome.Hsapiens.UCSC.hg19::Hsapiens)}

\paragraph{\texorpdfstring{\textbf{Arguments
}}{Arguments }}\label{arguments-10}

\begin{itemize}
\tightlist
\item
  \protect\hyperlink{vcfToSNV}{\texttt{snv}}: A dataframe having
  \texttt{sample,\ chr,\ pos,\ ref,\ alt\ and/or\ freq} as its
  columns.This dataframe may have more than one sample.
\item
  \texttt{th\_vec\_lw}: A vector of lower limits of frequency ranges.
\item
  \texttt{th\_vec\_up}: A vector of upper limits of frequency ranges.
  \texttt{signatures.ref}: An object of class mutsig comprising the set of signatures. (`signatures.nature2013' or `signatures.cosmic' or `signatures.cosmic.2019' ), Default: \href{https://cran.r-project.org/web/packages/deconstructSigs/deconstructSigs.pdf}{signatures.cosmic} 
\end{itemize}

\paragraph{\texorpdfstring{\textbf{Value}}{Value}}\label{value-9}

A list of samples having mutational signature corresponding to each allele frequency range

\paragraph{\texorpdfstring{\textbf{Examples}}{Examples}}\label{examples-11}

\begin{verbatim}
> data(snv_sample)   # load 'snv' dataframe object
> mut_sig_per_freq_range=persistSig(snv=snv_sample,th_vec_lw=c(0,0.4), th_vec_up=c(0.1,1), 
bsg = BSgenome.Hsapiens.UCSC.hg19::Hsapiens)  # list of samples having mutational signature for each allele frequency range i.e., in this example frequency ranges are (0.0 - 0.1) & (0.4 - 1.0).

> mut_sig_per_freq_range$CRF004556[,1:6] # view some signatures at different allele freq ranges
           sig.1 sig.2     sig.3 sig.4      sig.5      sig.6
0-0.1 0.02644092     0 0.6242417     0 0.00000000 0.00000000
0.4-1 0.19941897     0 0.2126304     0 0.09984891 0.08473515
\end{verbatim}

\subsubsection{3.14 enrichSig}\label{enrichsig}

\paragraph{\texorpdfstring{\textbf{Description}}{Description}}\label{description-10}

Determine over-represented mutation signatures in individual case
sample(s) relative to that in the panel of control samples. This is
suitable when case samples are heterogeneous, and only some of them
might have excess of certain mutation signatures of interest relative to
the control population.

\paragraph{\texorpdfstring{\textbf{Usage}}{Usage}}\label{usage-12}

\texttt{enrichSig(contextfreq.cases,\ contextfreq.controls,\ signatures.ref=signatures.cosmic,\ threshold=0.05)}

\paragraph{\texorpdfstring{\textbf{Arguments
}}{Arguments }}\label{arguments-11}

\begin{itemize}
\item
  \texttt{contextfreq.cases}: A data frame of class
  \protect\hyperlink{contextfreq}{\texttt{contextfreq}} containing
  mutation frequency in tri-nucleotide contexts in case samples
\item
  \texttt{contextfreq.controls}: A data frame of class
  \protect\hyperlink{contextfreq}{\texttt{contextfreq}} containing
  mutation frequency in tri-nucleotide contexts in control samples.
\item
  \texttt{signatures.ref}: An object of class mutsig comprising the set of signatures. (`signatures.nature2013' or `signatures.cosmic' or `signatures.cosmic.2019' ), Default: \href{https://cran.r-project.org/web/packages/deconstructSigs/deconstructSigs.pdf}{signatures.cosmic} 
\item
  \texttt{threshold}: Threshold for uncorrected percentile score.
  \texttt{Default:\ 0.05}
\end{itemize}

\paragraph{\texorpdfstring{\textbf{Details}}{Details}}\label{details-7}

Determine over-represented mutation signatures in individual case
sample(s), highlighting those that are significantly enriched. The
extent of enrichment is indicated using a percentile score, with low
scores indicating high enrichment for specific mutation signatures in a
case sample relative to that in the panel of control samples.

\paragraph{\texorpdfstring{\textbf{Value}}{Value}}\label{value-10}

An object of class \texttt{enrichSig.obj} providing the following
information:

\begin{itemize}
\tightlist
\item
  \texttt{n.case}: Number of case samples.
\item
  \texttt{n.control}: Number of control samples.
\item
  \texttt{case.weights}: A data frame containing estimated weights of
  known mutation signatures in the case samples.
\item
  \texttt{control.weights}: A data frame containing estimated weights of
  known mutation signatures in the control samples.
\item
  \texttt{case.percentile}: Percentile scores corresponding to the
  extent of enrichment of known mutation signatures in case sample(s)
  relative to that in the control samples.
\end{itemize}

\paragraph{\texorpdfstring{\textbf{Details}}{Details}}\label{details-8}

Determine over-represented mutation signatures in individual case
sample(s), highlighting those that are significantly enriched. The
extent of enrichment is indicated using a percentile score, with low
scores indicating high enrichment for specific mutation signatures in a
case sample relative to that in the panel of control samples.

\paragraph{\texorpdfstring{\textbf{Examples}}{Examples}}\label{examples-12}

\begin{verbatim}
> data(contextfreq.cases_test)
> data(contextfreq.controls_test)
> enrich_obj=enrichSig(contextfreq.case=contextfreq.cases_test,
	contextfreq.controls=contextfreq.controls_test, 
	signatures.ref=signatures.cosmic, threshold=0.05)
\end{verbatim}

\subsubsection{3.15 caseControlSig}\label{casecontrolsig}

\paragraph{\texorpdfstring{\textbf{Description}}{Description}}\label{description-11}

Identifies signatures with significantly higher mutation burden in case
samples over control samples.

\paragraph{\texorpdfstring{\textbf{Usage}}{Usage}}\label{usage-13}

\texttt{caseControlSig(contextfreq.cases,\ contextfreq.controls,\ signatures.ref=signatures.cosmic,\ threshold=0.05,\ adjust="fdr")}

\paragraph{\texorpdfstring{\textbf{Arguments
}}{Arguments }}\label{arguments-12}

\begin{itemize}
\tightlist
\item
  \texttt{contextfreq.cases}: A data frame of class
  \protect\hyperlink{contextfreq}{\texttt{contextfreq}} containing
  mutation frequency in tri-nucleotide contexts in case samples
\item
  \texttt{contextfreq.controls}: A data frame of class
  \protect\hyperlink{contextfreq}{\texttt{contextfreq}} containing
  mutation frequency in tri-nucleotide contexts in control samples.
\item
  \texttt{signatures.ref}: An object of class mutsig comprising the set of signatures. (`signatures.nature2013' or `signatures.cosmic' or `signatures.cosmic.2019' ), Default: \href{https://cran.r-project.org/web/packages/deconstructSigs/deconstructSigs.pdf}{signatures.cosmic} 
\item
  \texttt{threshold}: Threshold for uncorrected percentile score.
  \texttt{Default:\ 0.05}
\item
  \texttt{adjust}: Method for p-value correction for multiple testing.
  Options are as provided in the function \texttt{p.adjust}. The default
  method is the \texttt{FDR} method.
\end{itemize}

\paragraph{\texorpdfstring{\textbf{Details}}{Details}}\label{details-9}

Identifies signatures with significantly higher mutation burden in case
samples compared to control samples using the
\href{https://www.rdocumentation.org/packages/stats/versions/3.5.2/topics/wilcox.test}{Wilcoxon
Rank sum test}. This is suitable when the samples are homogeneous within
respective groups, and signatures significantly over-represented in
cases relative to the controls are of interest.

\paragraph{\texorpdfstring{\textbf{Value}}{Value}}\label{value-11}

An object providing the following information:

\begin{itemize}
\tightlist
\item
  \texttt{n.case} : Number of case samples.
\item
  \texttt{n.control}: Number of control samples.
\item
  \texttt{case.weights}: A data frame containing estimated weights of
  known mutation signatures in the case samples.
\item
  \texttt{control.weights}: A data frame containing estimated weights of
  known mutation signatures in the control samples.
\item
  \texttt{p.value} : Indicates statistical significance of higher
  mutation burden of signatures in cases over controls.
\item
  \texttt{adjust.p.value}: P-values adjusted for multiple testing
  correction.
\end{itemize}

\paragraph{\texorpdfstring{\textbf{Examples}}{Examples}}\label{examples-13}

\begin{verbatim}
> data(contextfreq.cases_test)
> data(contextfreq.controls_test)
> signif_signatures_obj=caseControlSig(contextfreq.cases=contextfreq.cases_test, 
contextfreq.controls=contextfreq.controls_test, signatures.ref=signatures.cosmic, 
threshold=0.05, adjust="fdr")

\end{verbatim}

\bibliography{bibliography.bib}

\printindex

\end{document}
